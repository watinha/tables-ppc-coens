
\begin{quadro}[ht!]
  \centering
\caption{Unidade Curricular Algoritmos 1}
\label{unit_themes_ras_0}
\begin{tabular}{|p{5cm}|p{8cm}|}\hline
{\cellcolor{blue1} Unidade curricular} & Algoritmos 1\\\hline
{\cellcolor{blue1} Núcleo} & Básico\\\hline
{\cellcolor{blue1} Carga-horária (Horas)} & 90\\\hline
\multicolumn{2}{|p{13cm}|}{Temas de Estudo} \\\hline
\multicolumn{2}{|p{13cm}|}{\xitem TE1: Computadores: organização, características, funcionamento, descrição narrativa e fluxogramas (10h)\\\xitem TE2: Algoritmos: tipos primitivos de dados, constantes e variáveis, expressões lógicas e aritméticas e práticas com programação (10h)\\\xitem TE3: Estrutura de controle: desvio condicional e laços de repetição e práticas com programação (20h)\\\xitem TE4: Modularização: funções, procedimentos, passagem de parâmetros e práticas com programação (20h)\\\xitem TE5: Vetores, matrizes e registros: definições, manipulação e práticas com programação (20h)\\\xitem TE6: Fundamentos de teste de software: teste de mesa e depuração (10h)} \\\hline
\multicolumn{2}{|p{13cm}|}{Resultados de Aprendizagem} \\\hline
\multicolumn{2}{|p{13cm}|}{\xitem RA1: Identificar constantes e variáveis de entrada e saída considerando as características e organização de computadores utilizando tipos primitivos de dados (T1 e T2).\\\xitem RA2: Criar algoritmos para os problemas delimitados utilizando descrição narrativa, fluxogramas, estruturas de controle e expressões lógicas e aritméticas (T2 e T3).\\\xitem RA3: Implementar programas para solucionar os problemas estruturados utilizando tipos primitivos de dados, vetores, matrizes e registros, por meio de linguagem de programação (T3 e T4)\\\xitem R4: Validar a solução proposta por meio da execução do programa implementado verificando se as variáveis de entrada e saída condizem com o esperado. (T1, T2 e T3)\\\xitem RA5: Recriar algoritmos quando a solução proposta não for validada, utilizando testes de mesa e depuração de software. (T6)} \\\hline

	\end{tabular}
\end{quadro}
\begin{quadro}[ht!]
  \centering
\caption{Unidade Curricular Introdução a Engenharia de Software}
\label{unit_themes_ras_1}
\begin{tabular}{|p{5cm}|p{8cm}|}\hline
{\cellcolor{blue1} Unidade curricular} & Introdução a Engenharia de Software\\\hline
{\cellcolor{blue1} Núcleo} & Básico\\\hline
{\cellcolor{blue1} Carga-horária (Horas)} & 30\\\hline
\multicolumn{2}{|p{13cm}|}{Temas de Estudo} \\\hline
\multicolumn{2}{|p{13cm}|}{\xitem TE1: Profissional de Engenharia de Software: riscos no ambiente de trabalho (incêndio e desastres), visão geral do profissional, áreas de atuação e relações com o perfil do egresso do curso da UTFPR. (8h)\\\xitem TE2: Aspectos da profissão: desafios do profissional, conceitos de ética e código de ética profissional (8h)\\\xitem TE3: Visão de mercado: novas demandas e impactos no profissional, interdisciplinariedade e a aplicação em dimensões social, política, econômica, cultural e ambiental (14h)} \\\hline
\multicolumn{2}{|p{13cm}|}{Resultados de Aprendizagem} \\\hline
\multicolumn{2}{|p{13cm}|}{\xitem RA1: Identificar contextos sociais de aplicação e áreas de atuação do profissional egresso do curso, considerando requisitos da profissão, responsabilidades e ética (TE1 e TE2).\\\xitem RA2: Analisar criticamente a aplicação de TI considerando problemáticas de dimensões social, política, econômica, cultural e ambiental, de forma crítica e criativa (TE3).} \\\hline

	\end{tabular}
\end{quadro}
\begin{quadro}[ht!]
  \centering
\caption{Unidade Curricular Organização de Computadores}
\label{unit_themes_ras_2}
\begin{tabular}{|p{5cm}|p{8cm}|}\hline
{\cellcolor{blue1} Unidade curricular} & Organização de Computadores\\\hline
{\cellcolor{blue1} Núcleo} & Básico\\\hline
{\cellcolor{blue1} Carga-horária (Horas)} & 30\\\hline
\multicolumn{2}{|p{13cm}|}{Temas de Estudo} \\\hline
\multicolumn{2}{|p{13cm}|}{\xitem TE1: Organização de computadores: hierarquia de memória e troca de dados, memória virtual, componentes da unidade central de processamento, barramentos e dispositivos de entrada e saída (15h)\\\xitem TE2: Operações: interrupções, exceções, operações de entrada e saída, hierarquia de barramentos (7,5h)\\\xitem TE3: Pipeline: ciclo de instrução, funcionamento e problemas (7,5h)} \\\hline
\multicolumn{2}{|p{13cm}|}{Resultados de Aprendizagem} \\\hline
\multicolumn{2}{|p{13cm}|}{\xitem RA1: Compreender a estrutura básica de funcionamento de computadores, aplicando fundamentos matemáticos, técnicos e tecnológicos (TE1 e TE2).\\\xitem RA2: Elaborar hipóteses sobre a ordem de execução de instruções no processador utilizando tratamento de interrupções, exceções e pipeline (TE2 e TE3).} \\\hline

	\end{tabular}
\end{quadro}
\begin{quadro}[ht!]
  \centering
\caption{Unidade Curricular Fundamentos de Matemática}
\label{unit_themes_ras_3}
\begin{tabular}{|p{5cm}|p{8cm}|}\hline
{\cellcolor{blue1} Unidade curricular} & Fundamentos de Matemática\\\hline
{\cellcolor{blue1} Núcleo} & Básico\\\hline
{\cellcolor{blue1} Carga-horária (Horas)} & 60\\\hline
\multicolumn{2}{|p{13cm}|}{Temas de Estudo} \\\hline
\multicolumn{2}{|p{13cm}|}{\xitem TE1: Conjuntos numéricos e intervalos: operações e representações (12h)\\\xitem TE2: Álgebra básica: operações com polinômios, produtos, notáveis, propriedades da potenciação e radiciação, equações e inequações (métodos de resolução e validação) (12h)\\\xitem TE3: Funções: conceitos, tipos de função representação algébrica, e gráfica com o auxílio de software classificações e aplicações na engenharia (16h)\\\xitem TE4: Trigonometria no triângulo retângulo e no ciclo trigonométrico: razões trigonométricas no triângulo retângulo, medidas de ângulos em graus e radianos, o ciclo trigonométrico, funções trigonométricas e suas inversas e aplicações à engenharia (12h)\\\xitem TE5: Números complexos: Conceito, representação na forma algébrica e operações, representação gráfica, representação polar e operações (6h)} \\\hline
\multicolumn{2}{|p{13cm}|}{Resultados de Aprendizagem} \\\hline
\multicolumn{2}{|p{13cm}|}{\xitem RA1: Interpretar problemas estruturados representando-os adequadamente por meio de conjuntos, expressões algébricas, polinômios e números complexos (TE1, TE2 e TE5).\\\xitem RA2: Representar limites dos problemas corretamente por meio de equações, inequações e funções identificando variáveis de decisão e domínio do problema (TE3 e TE4).\\\xitem RA3: Resolver problemas delimitados, empregando corretamente equações, inequações e funções (TE3 e TE4).} \\\hline

	\end{tabular}
\end{quadro}
\begin{quadro}[ht!]
  \centering
\caption{Unidade Curricular Comunicação Oral e Escrita}
\label{unit_themes_ras_4}
\begin{tabular}{|p{5cm}|p{8cm}|}\hline
{\cellcolor{blue1} Unidade curricular} & Comunicação Oral e Escrita\\\hline
{\cellcolor{blue1} Núcleo} & Básico\\\hline
{\cellcolor{blue1} Carga-horária (Horas)} & 60\\\hline
\multicolumn{2}{|p{13cm}|}{Temas de Estudo} \\\hline
\multicolumn{2}{|p{13cm}|}{\xitem TE1: Estratégias de leitura acadêmica: antecipação, seleção e síntese de informações em artigos científicos. (10h)\\\xitem TE2: Processo de produção de resumo e resenha acadêmicos: diferença, estrutura composicional, planejamento, escrita e revisão (10h)\\\xitem TE3: Processo de produção de gêneros orais acadêmicos: planejamento, produção de materiais de apoio e técnicas de interação com o público. (10h)\\\xitem TE4: Processo de produção de gêneros orais empresariais: diretrizes da comunicação na organização, estratégias para comunicação externa e interna e o evento reunião empresarial. (15h)\\\xitem TE5: Processo de comunicação para apresentação em público: elementos da comunicação verbal e não verbal e procedimentos para apresentação, elaboração e avaliação de exposição oral. (15h)} \\\hline
\multicolumn{2}{|p{13cm}|}{Resultados de Aprendizagem} \\\hline
\multicolumn{2}{|p{13cm}|}{\xitem RA1: Analisar criticamente informações e hipóteses validadas, com responsabilidade, de forma autônoma e colaborativa, empregando estratégias de leitura acadêmica (TE1).\\\xitem RA2: Aplicar soluções tecnológicas a partir da produção de gêneros orais e escritos acadêmicos e comunicação qualificada, adequada e assertiva (TE2 e TE3)\\\xitem RA3: Planejar a apresentação pessoal em reunião empresarial, com ética, criticidade e assertividade, conforme as diretrizes de comunicação da organização e empregando estratégias de comunicação interna (TE4)\\\xitem RA4: Expor oralmente temáticas da área de comunicação empresarial, avaliando o próprio desempenho e o de colegas, conforme o plano proposto, de forma coerente, autônoma e criativa (TE5)} \\\hline

	\end{tabular}
\end{quadro}
\begin{quadro}[ht!]
  \centering
\caption{Unidade Curricular Inglês Instrumental}
\label{unit_themes_ras_5}
\begin{tabular}{|p{5cm}|p{8cm}|}\hline
{\cellcolor{blue1} Unidade curricular} & Inglês Instrumental\\\hline
{\cellcolor{blue1} Núcleo} & Básico\\\hline
{\cellcolor{blue1} Carga-horária (Horas)} & 30\\\hline
\multicolumn{2}{|p{13cm}|}{Temas de Estudo} \\\hline
\multicolumn{2}{|p{13cm}|}{\xitem TE1 - Estratégias de leitura: definições e prática de estratégias pre-reading (skimming e scanning) e reading. (10h)\\\xitem TE2: Principais gêneros das esferas instrucional e virtual: definições, tipologia textual e gêneros de aplicação em TADS (tutoriais, normas, blogs, fóruns etc.). (10h)\\\xitem TE3 - Estratégias de leitura frasal da língua inglesa: cognatos e principais estruturas gramaticais. (10h)} \\\hline
\multicolumn{2}{|p{13cm}|}{Resultados de Aprendizagem} \\\hline
\multicolumn{2}{|p{13cm}|}{\xitem RA1: Analisar textos em inglês de diferentes gêneros, aplicando estratégias de leitura para o desenvolvimento da autonomia e responsabilidade nas análises efetuadas (TE1)\\\xitem RA2: Identificar os principais gêneros das esferas instrucionais e virtual, aplicando o conhecimento sobre as características presentes em diferentes gêneros textuais (TE2).\\\xitem RA3: Deduzir significados a partir da semelhança entre as palavras na língua portuguesa e inglesa, aplicando os conhecimentos de cognatos com autonomia (TE3)} \\\hline

	\end{tabular}
\end{quadro}
\begin{quadro}[ht!]
  \centering
\caption{Unidade Curricular Algoritmos 2}
\label{unit_themes_ras_6}
\begin{tabular}{|p{5cm}|p{8cm}|}\hline
{\cellcolor{blue1} Unidade curricular} & Algoritmos 2\\\hline
{\cellcolor{blue1} Núcleo} & Básico\\\hline
{\cellcolor{blue1} Carga-horária (Horas)} & 60\\\hline
\multicolumn{2}{|p{13cm}|}{Temas de Estudo} \\\hline
\multicolumn{2}{|p{13cm}|}{\xitem TE1: Ponteiros: declaração, uso e atribuição, operações, representação de ponteiros na memória, passagem por referência (15h)\\\xitem TE2: Alocação Dinâmica: alocação de espaço de memória, operações com memória alocada, ponteiros de ponteiros, exemplos vetores, matrizes e registros (20h)\\\xitem TE3: Recursividade: noções de iteração com recursividade, mapeamento de laços para recursividade, recursividade com retorno, recursividade múltipla, exemplo de otimização de recursividade múltipla via programação dinâmica (25h)} \\\hline
\multicolumn{2}{|p{13cm}|}{Resultados de Aprendizagem} \\\hline
\multicolumn{2}{|p{13cm}|}{\xitem RA1: Identificar problemas de algoritmos que se beneficiam do uso de ponteiros para gerar códigos eficientes (TE1).\\\xitem RA2: Analisar a aplicação de tecnologias da engenharia de computação considerando códigos de ética e a postura profissional dos(as) engenheiros(as) (TE3).\\\xitem RA3: Replanejar a implementação de algoritmos utilizando o conceito de recursividade para problemas que envolvam a execução de laços e iteratividade (TE3).} \\\hline

	\end{tabular}
\end{quadro}
\begin{quadro}[ht!]
  \centering
\caption{Unidade Curricular Processo de Produção de Software}
\label{unit_themes_ras_7}
\begin{tabular}{|p{5cm}|p{8cm}|}\hline
{\cellcolor{blue1} Unidade curricular} & Processo de Produção de Software\\\hline
{\cellcolor{blue1} Núcleo} & Específico\\\hline
{\cellcolor{blue1} Carga-horária (Horas)} & 30\\\hline
\multicolumn{2}{|p{13cm}|}{Temas de Estudo} \\\hline
\multicolumn{2}{|p{13cm}|}{\xitem TE1: Modelos de processo de desenvolvimento de software: etapas do processo, áreas de conhecimento da Engenharia de Software, modelos cascata, sequencial, espiral. (10h)\\\xitem TE2: Metodologias ágeis: manifesto ágil, Scrum e XP. (20h)} \\\hline
\multicolumn{2}{|p{13cm}|}{Resultados de Aprendizagem} \\\hline
\multicolumn{2}{|p{13cm}|}{\xitem RA1: Compreender os modelos de processo de desenvolvimento de software (TE1)\\\xitem RA2: Aplicar práticas de metodologias ágeis no desenvolvimento de projetos (TE2)} \\\hline

	\end{tabular}
\end{quadro}
\begin{quadro}[ht!]
  \centering
\caption{Unidade Curricular Sistemas Operacionais}
\label{unit_themes_ras_8}
\begin{tabular}{|p{5cm}|p{8cm}|}\hline
{\cellcolor{blue1} Unidade curricular} & Sistemas Operacionais\\\hline
{\cellcolor{blue1} Núcleo} & Básico\\\hline
{\cellcolor{blue1} Carga-horária (Horas)} & 60\\\hline
\multicolumn{2}{|p{13cm}|}{Temas de Estudo} \\\hline
\multicolumn{2}{|p{13cm}|}{\xitem TE1: Sistemas operacionais: classificação, estrutura, componentes, concorrência, configuração e utilização de sistemas Linux (15h)\\\xitem TE2: Processos: definição e exemplos dos uso de threads, escalonamento, sincronização e deadlocks (15h)\\\xitem TE3: Memória: formas de endereçamento, gerenciamento de memória virtual e real, alocação, swapping, segmentação e paginação de memória (15h)\\\xitem TE4: Gerenciamento de dispositivos de entrada e saída: discos, interfaces com o usuário e o conceito de spooling (15h)} \\\hline
\multicolumn{2}{|p{13cm}|}{Resultados de Aprendizagem} \\\hline
\multicolumn{2}{|p{13cm}|}{\xitem RA1: Gerenciar serviços de sistemas operacionais, inicializando, ajustando permissões, analisando recursos de entrada/saída, memória e processamento, automatizando tarefas e finalizando processos com eficácia (TE1, TE3 e TE4).\\\xitem RA2: Implementar aplicações com diferentes processos e threads explorando sua sincronização e evitando a ocorrência de deadlocks (TE2).\\\xitem RA3: Implementar sistemas de software com uso de chamadas de sistema do Linux de entrada e saída, controle de processos e gerenciamento de arquivos (TE1, TE2 e TE4).\\\xitem RA4: Reconfigurar serviços de sistemas operacionais a partir do balanceamento de consumo de recursos do sistema (TE1, TE3 e TE4).} \\\hline

	\end{tabular}
\end{quadro}
\begin{quadro}[ht!]
  \centering
\caption{Unidade Curricular Banco de Dados}
\label{unit_themes_ras_9}
\begin{tabular}{|p{5cm}|p{8cm}|}\hline
{\cellcolor{blue1} Unidade curricular} & Banco de Dados\\\hline
{\cellcolor{blue1} Núcleo} & Básico\\\hline
{\cellcolor{blue1} Carga-horária (Horas)} & 90\\\hline
\multicolumn{2}{|p{13cm}|}{Temas de Estudo} \\\hline
\multicolumn{2}{|p{13cm}|}{\xitem TE1: Princípios de banco de dados: conceitos, aplicações e modelo relacional (relação, tupla, atributo, restrições de integridade e normalização) e construção de representações. (15h)\\\xitem TE2: Linguagem de Definição de Dados (DDL): criação, atualização e regras de integridade. (10h)\\\xitem TE3: Linguagem de Manipulação de Dados (DML): consultas simples, consultas aninhadas, operações de conjuntos, funções de agregação, inserção, remoção e atualização. (15h)\\\xitem TE4: Projeto de Banco de Dados: projeto conceitual, projeto lógico e projeto físico. (20h)} \\\hline
\multicolumn{2}{|p{13cm}|}{Resultados de Aprendizagem} \\\hline
\multicolumn{2}{|p{13cm}|}{\xitem RA1: Modelar esquemas de banco de dados, de forma autorregulada, interpretando problemas de armazenamento de dados estruturados (TE1).\\\xitem RA2: Implementar esquemas de banco de dados de acordo com padrões de codificação da linguagem de consulta estruturada SQL (TE2 e TE3).\\\xitem RA3: Projetar esquemas de banco de dados utilizando ferramentas de modelagem e SQL, com objetividade e clareza (TE4).} \\\hline

	\end{tabular}
\end{quadro}
\begin{quadro}[ht!]
  \centering
\caption{Unidade Curricular Fundamentos de Ética}
\label{unit_themes_ras_10}
\begin{tabular}{|p{5cm}|p{8cm}|}\hline
{\cellcolor{blue1} Unidade curricular} & Fundamentos de Ética\\\hline
{\cellcolor{blue1} Núcleo} & Básico\\\hline
{\cellcolor{blue1} Carga-horária (Horas)} & 30\\\hline
\multicolumn{2}{|p{13cm}|}{Temas de Estudo} \\\hline
\multicolumn{2}{|p{13cm}|}{\xitem TE1: Ética e Moral: aproximações (5h)\\\xitem TE2: Períodos e projetos ético-filosóficos: a Ética em Sócrates, Platão, Aristóteles, no Estoicismo, Epicurismo, Medievo e Humanismo, em Immanuel Kant (Dever), Arthur Schopenhauer (Compaixão), Friedrich Nietzsche (Moral Aristocrática X Moral de Rebanho) e Jean-Paul Sartre (Liberdade e Responsabilidade) (15h)\\\xitem TE3: Ética na Engenharia e em Computação: legislação (códigos de Ética), normas e postura profissional dos/as engenheiros/as (10h)} \\\hline
\multicolumn{2}{|p{13cm}|}{Resultados de Aprendizagem} \\\hline
\multicolumn{2}{|p{13cm}|}{\xitem RA1: Compreender diferentes períodos e projetos ético-filosóficos para considerar seus conceitos na análise do problema estático\\\xitem estruturado (TE1 e TE2).\\\xitem RA2: Analisar a aplicação de tecnologias da engenharia de computação considerando códigos de ética e a postura profissional dos(as) engenheiros(as) (TE3).} \\\hline

	\end{tabular}
\end{quadro}
\begin{quadro}[ht!]
  \centering
\caption{Unidade Curricular Matemática Discreta}
\label{unit_themes_ras_11}
\begin{tabular}{|p{5cm}|p{8cm}|}\hline
{\cellcolor{blue1} Unidade curricular} & Matemática Discreta\\\hline
{\cellcolor{blue1} Núcleo} & Básico\\\hline
{\cellcolor{blue1} Carga-horária (Horas)} & 60\\\hline
\multicolumn{2}{|p{13cm}|}{Temas de Estudo} \\\hline
\multicolumn{2}{|p{13cm}|}{\xitem TE1: Teoria dos conjuntos: conjuntos, lógica, funções e relações. (15h)\\\xitem TE2: Teoria dos números: números inteiros, princípios de indução e outras provas matemáticas. (15h)\\\xitem TE3: Combinatória: regras básicas de contagem, princípio de inclusão e exclusão. (15h)\\\xitem TE4: Grafos: árvores, fluxos em redes, emparelhamentos, grafos eulerianos,\\\xitem hamiltonianos, planares e coloridos. (15h)} \\\hline
\multicolumn{2}{|p{13cm}|}{Resultados de Aprendizagem} \\\hline
\multicolumn{2}{|p{13cm}|}{\xitem RA1: Resolver problemas que envolvam lógica simples, considerando conjuntos, funções e relações (TE1).\\\xitem RA2: Entender princípios de números inteiros e provas por direta, contraposição, absurdo e indução (TE2).\\\xitem RA3: Comparar dimensões de funções matemáticas, utilizando análise combinatória (TE3).\\\xitem RA4: Modelar problemas matemáticos utilizando diferentes conceitos de grafos (TE4).} \\\hline

	\end{tabular}
\end{quadro}
\begin{quadro}[ht!]
  \centering
\caption{Unidade Curricular Estrutura de Dados 1}
\label{unit_themes_ras_12}
\begin{tabular}{|p{5cm}|p{8cm}|}\hline
{\cellcolor{blue1} Unidade curricular} & Estrutura de Dados 1\\\hline
{\cellcolor{blue1} Núcleo} & Básico\\\hline
{\cellcolor{blue1} Carga-horária (Horas)} & 90\\\hline
\multicolumn{2}{|p{13cm}|}{Temas de Estudo} \\\hline
\multicolumn{2}{|p{13cm}|}{\xitem TE1: Tipos abstratos de dados: definições, tipos e recursos para implementações (8h)\\\xitem TE2: Pilhas e Filas: definições e operações, implementações e aplicações (16h)\\\xitem TE3: Listas: definição, tipos, operações, implementação e aplicações (16h)\\\xitem TE4: Árvore Binária: operações, busca e aplicações (20h)} \\\hline
\multicolumn{2}{|p{13cm}|}{Resultados de Aprendizagem} \\\hline
\multicolumn{2}{|p{13cm}|}{\xitem RA1: Identificar diferentes tipos abstratos de dados considerando suas características e recursos para implementações (TE1).\\\xitem RA2: Implementar estruturas de dados de pilhas, filas e listas identificando cenários de aplicações de maneira eficiente (TE2 e TE3).\\\xitem RA3: Aplicar os algoritmos de árvores para melhorar a eficiência de busca de maneira pró-ativa (TE3).} \\\hline

	\end{tabular}
\end{quadro}
\begin{quadro}[ht!]
  \centering
\caption{Unidade Curricular Programação Orientada a Objetos 1 }
\label{unit_themes_ras_13}
\begin{tabular}{|p{5cm}|p{8cm}|}\hline
{\cellcolor{blue1} Unidade curricular} & Programação Orientada a Objetos 1 \\\hline
{\cellcolor{blue1} Núcleo} & Específico\\\hline
{\cellcolor{blue1} Carga-horária (Horas)} & 90\\\hline
\multicolumn{2}{|p{13cm}|}{Temas de Estudo} \\\hline
\multicolumn{2}{|p{13cm}|}{\xitem TE1: Fundamentos de orientação a objetos: definições, artefatos, atributos, métodos, objetos, tipos de classes, operador de casting e o paralelo entre orientação à objetos e o paradigma estruturado (20h)\\\xitem TE2: Relacionamento entre classes e objetos: encapsulamento, herança, tipos de polimorfismo e interfaces (15h)\\\xitem TE3: Tratamento de exceções: definição, mecanismos de disparo e tratamento e classes de exceção (10h)\\\xitem TE4: Desenvolvimento de aplicações: utilização de IDEs para implementação de aplicações utilizando paradigma de orientação à objetos (15h)} \\\hline
\multicolumn{2}{|p{13cm}|}{Resultados de Aprendizagem} \\\hline
\multicolumn{2}{|p{13cm}|}{\xitem RA1: Implementar algoritmos utilizando classes, atributos, objetos e métodos com autorregulação (TE1).\\\xitem RA2: Selecionar o uso de encapsulamento, polimorfismo e herança para representar relacionamentos entre classes (TE2).\\\xitem RA3: Identificar exceções mais relevantes para o tratamento no contexto do paradigma de orientação a objetos (TE3).\\\xitem RA4: Implementar aplicações utilizando uma linguagem de programação orientada a objeto com clareza (TE4).} \\\hline

	\end{tabular}
\end{quadro}
\begin{quadro}[ht!]
  \centering
\caption{Unidade Curricular Requisitos de Software}
\label{unit_themes_ras_14}
\begin{tabular}{|p{5cm}|p{8cm}|}\hline
{\cellcolor{blue1} Unidade curricular} & Requisitos de Software\\\hline
{\cellcolor{blue1} Núcleo} & Específico\\\hline
{\cellcolor{blue1} Carga-horária (Horas)} & 60\\\hline
\multicolumn{2}{|p{13cm}|}{Temas de Estudo} \\\hline
\multicolumn{2}{|p{13cm}|}{\xitem TE1: Requisitos de Software: tipos de requisitos (funcionais/não funcionais), níveis de requisitos (do usuário, do sistema e de domínio) e processo de Engenharia de Requisitos (estudo de viabilidade, elicitação/ análise (técnicas de levantamento), especificação/ documentação e validação de requisitos. (20h)\\\xitem TE2: Modelagem conceitual: modelos estruturais e comportamentais da UML (Diagrama de Casos de Uso, Diagrama de atividades, Diagrama de classes) e Ferramentas de Modelagem. (20h)\\\xitem TE3: Gerência de Requisitos: Controle de mudança, Controle de versão e rastreabilidade de requisitos. (10h)\\\xitem TE4: Engenharia de Requisitos em projetos ágeis: elicitação de requisitos (reuniões, brainstorming, JAD), especificação de requisitos (histórias de usuário, use case, protótipos e cartão de história) e gerenciamento de requisitos em projetos ágeis (controle de mudança e pessoas). (10h)} \\\hline
\multicolumn{2}{|p{13cm}|}{Resultados de Aprendizagem} \\\hline
\multicolumn{2}{|p{13cm}|}{\xitem RA1: Aplicar o processo de engenharia de requisitos considerando diferentes níveis e tipos de requisitos com eficácia, objetividade e clareza (TE1).\\\xitem RA2: Projetar sistemas computacionais utilizando modelagem conceitual e ferramentas com eficácia, objetividade e clareza (TE2).\\\xitem RA3: Utilizar práticas de controle de mudança para gerenciar requisitos em projetos tradicionais e ágeis utilizando ferramentas e novas tecnologias (TE3 e TE4).} \\\hline

	\end{tabular}
\end{quadro}
\begin{quadro}[ht!]
  \centering
\caption{Unidade Curricular Probabilidade e Estatística}
\label{unit_themes_ras_15}
\begin{tabular}{|p{5cm}|p{8cm}|}\hline
{\cellcolor{blue1} Unidade curricular} & Probabilidade e Estatística\\\hline
{\cellcolor{blue1} Núcleo} & Básico\\\hline
{\cellcolor{blue1} Carga-horária (Horas)} & 60\\\hline
\multicolumn{2}{|p{13cm}|}{Temas de Estudo} \\\hline
\multicolumn{2}{|p{13cm}|}{\xitem TE1: Probabilidade: espaço amostral, eventos, axiomas e teoremas (10h)\\\xitem TE2: Variável aleatórias: variável aleatória discreta, variável aleatória contínua e função de probabilidade (10h)\\\xitem TE3: Inferência estatística: estimação de parâmetros, intervalos de confiança e testes de hipóteses (10h)\\\xitem TE4: Testes de hipóteses: tipos de testes e aplicações (20h)\\\xitem TE5: Controle Estatístico do Processo: diagrama de controle e aplicações (10h)} \\\hline
\multicolumn{2}{|p{13cm}|}{Resultados de Aprendizagem} \\\hline
\multicolumn{2}{|p{13cm}|}{\xitem RA1: Compreender os conceitos da teoria das probabilidades identificando as funcionalidades dos mesmos na estrutura e na modelagem probabilística de dados e experimentos (TE1 e TE2).\\\xitem RA2: Elaborar hipóteses e inferir características acerca de uma população de dados por meio de amostras utilizando-se de conceitos e métodos estatísticos (TE3).\\\xitem RA3: Solucionar problemas estáticos aplicando testes de hipóteses e empregando ferramentas tecnológicas (TE4).\\\xitem RA4: Identificar possíveis falhas de um processo por meio de métodos estatísticos, de forma sistemática (TE5).} \\\hline

	\end{tabular}
\end{quadro}
\begin{quadro}[ht!]
  \centering
\caption{Unidade Curricular Redes de Computadores}
\label{unit_themes_ras_16}
\begin{tabular}{|p{5cm}|p{8cm}|}\hline
{\cellcolor{blue1} Unidade curricular} & Redes de Computadores\\\hline
{\cellcolor{blue1} Núcleo} & Básico\\\hline
{\cellcolor{blue1} Carga-horária (Horas)} & 60\\\hline
\multicolumn{2}{|p{13cm}|}{Temas de Estudo} \\\hline
\multicolumn{2}{|p{13cm}|}{\xitem TE1: Princípios de Redes de Computadores e a Internet: conceitos, definições e aplicações, arquitetura de redes, topologia, arquitetura em camadas da Internet. (8h)\\\xitem TE2: Camada de Aplicação: conceitos e utilização dos protocolos HTTP, SSH, FTP, DHCP, SMTP, IMAP, POP3, DNS, P2P, VPN. (30h)\\\xitem TE3: Camada de Transporte e de Rede: TCP, UDP, IP, ICMP, Máscara de Sub-rede, Roteamento e IPSec. (14h)\\\xitem TE4: Camada de Enlace e Física: LLC, MAC, Equipamentos de conectividade e meios de transmissão. (8h)} \\\hline
\multicolumn{2}{|p{13cm}|}{Resultados de Aprendizagem} \\\hline
\multicolumn{2}{|p{13cm}|}{\xitem RA1: Selecionar protocolos da camada de aplicação adequados à resolução do problema de modo autorregulado (TE1 e TE2).\\\xitem RA2: Aplicar protocolos e configurações adequadas à resolução do problema por meio de raciocínio lógico e computacional de maneira autônoma e eficiente (TE3 e TE4).} \\\hline

	\end{tabular}
\end{quadro}
\begin{quadro}[ht!]
  \centering
\caption{Unidade Curricular Estrutura de Dados 2}
\label{unit_themes_ras_17}
\begin{tabular}{|p{5cm}|p{8cm}|}\hline
{\cellcolor{blue1} Unidade curricular} & Estrutura de Dados 2\\\hline
{\cellcolor{blue1} Núcleo} & Básico\\\hline
{\cellcolor{blue1} Carga-horária (Horas)} & 90\\\hline
\multicolumn{2}{|p{13cm}|}{Temas de Estudo} \\\hline
\multicolumn{2}{|p{13cm}|}{\xitem TE1: Algoritmos de ordenação: fundamentos, aplicações e complexidade computacional (12h)\\\xitem TE2: Árvore balanceada: operações, busca e aplicações de árvores avl e rubro-negra (14h)\\\xitem TE3: Árvore B: operações e aplicações (14h)\\\xitem TE4: Grafos: representação, operações, busca e aplicações (20h)} \\\hline
\multicolumn{2}{|p{13cm}|}{Resultados de Aprendizagem} \\\hline
\multicolumn{2}{|p{13cm}|}{\xitem RA1: Aplicar algoritmos de ordenação exercitando o raciocínio lógico de maneira eficiente (TE1).\\\xitem RA2: Realizar buscas em árvores balanceadas e grafos para encontrar soluções aos problemas modelados com essas estruturas de dados (TE2 e TE4).\\\xitem RA3: Replanejar a solução de problemas selecionando diferentes tipos de dados, como árvores avl, rubro-negra, árvore B ou grafos (TE2, TE3 e TE4).} \\\hline

	\end{tabular}
\end{quadro}
\begin{quadro}[ht!]
  \centering
\caption{Unidade Curricular Programação Orientada a Objetos 2}
\label{unit_themes_ras_18}
\begin{tabular}{|p{5cm}|p{8cm}|}\hline
{\cellcolor{blue1} Unidade curricular} & Programação Orientada a Objetos 2\\\hline
{\cellcolor{blue1} Núcleo} & Específico\\\hline
{\cellcolor{blue1} Carga-horária (Horas)} & 90\\\hline
\multicolumn{2}{|p{13cm}|}{Temas de Estudo} \\\hline
\multicolumn{2}{|p{13cm}|}{\xitem TE1: Boas práticas de programação: definições, manutenção, segurança e versionamento de software (10h)\\\xitem TE2: Linguagem de modelagem unificada (UML): definições, objetivos, exemplos e projeto de implementação com base em documentação (10h)\\\xitem TE3: Armazenamento persistente de dados: iteração com banco de dados a partir do paradigma de orientação a objetos (20h)\\\xitem TE4: Threads e concorrência: definições, prioridades, sincronização e aplicações de threads no contexto do mundo do trabalho (20h)} \\\hline
\multicolumn{2}{|p{13cm}|}{Resultados de Aprendizagem} \\\hline
\multicolumn{2}{|p{13cm}|}{\xitem RA1: Aplicar boas práticas de programação com ferramentas de versionamento de software para facilitar sua segurança e manutenção (TE1).\\\xitem RA2: Documentar o desenvolvimento de software com a linguagem de modelagem unificada (UML) com o intuito de facilitar futuras manutenções e torná-lo mais eficiente (TE2).\\\xitem RA3: Integrar aplicações desenvolvidas com o paradigma de orientação a objetos com bancos de dados para tornar o armazenamento persistente (TE3).\\\xitem RA4: Definir uso de threads e concorrência identificando possíveis problemas de sincronização e prioridades para diferentes aplicações no contexto do mundo do trabalho (TE4).} \\\hline

	\end{tabular}
\end{quadro}
\begin{quadro}[ht!]
  \centering
\caption{Unidade Curricular Sistemas Distribuídos}
\label{unit_themes_ras_19}
\begin{tabular}{|p{5cm}|p{8cm}|}\hline
{\cellcolor{blue1} Unidade curricular} & Sistemas Distribuídos\\\hline
{\cellcolor{blue1} Núcleo} & Específico\\\hline
{\cellcolor{blue1} Carga-horária (Horas)} & 60\\\hline
\multicolumn{2}{|p{13cm}|}{Temas de Estudo} \\\hline
\multicolumn{2}{|p{13cm}|}{\xitem TE1: Sistemas Distribuídos: introdução e aplicações (5h)\\\xitem TE2: Arquiteturas: tipos, middleware e autogerenciamento (10h)\\\xitem TE3: Comunicação: sockets, métodos remotos e mensageria (15h)\\\xitem TE4: Sincronização: relógios físicos e lógicos e exclusão mútua (10h)\\\xitem TE5: Replicação e Consistência: motivação, escalabilidade, modelos e protocolos (10h)\\\xitem TE6: Atuação no Projeto: estudo de caso, simulação e desenvolvimento de cenários (10h)} \\\hline
\multicolumn{2}{|p{13cm}|}{Resultados de Aprendizagem} \\\hline
\multicolumn{2}{|p{13cm}|}{\xitem RA1: Identificar arquiteturas e propostas baseadas nos requisitos das aplicações com eficiência (TE1 e TE2).\\\xitem RA2: Desenvolver aplicações distribuídas de acordo com o requisito de transparência da aplicação (TE3).\\\xitem RA3: Escolher propostas de sincronização e escalabilidade com eficiência (TE4 e TE5).\\\xitem RA4: Utilizar técnicas de desenvolvimento distribuído para aplicações escaláveis com eficiência (TE6).} \\\hline

	\end{tabular}
\end{quadro}
\begin{quadro}[ht!]
  \centering
\caption{Unidade Curricular Programação Web Front-end}
\label{unit_themes_ras_20}
\begin{tabular}{|p{5cm}|p{8cm}|}\hline
{\cellcolor{blue1} Unidade curricular} & Programação Web Front-end\\\hline
{\cellcolor{blue1} Núcleo} & Específico\\\hline
{\cellcolor{blue1} Carga-horária (Horas)} & 60\\\hline
\multicolumn{2}{|p{13cm}|}{Temas de Estudo} \\\hline
\multicolumn{2}{|p{13cm}|}{\xitem TE1: Desenvolvimento de aplicações para cliente na Web: tecnologias fundacionais e arquitetura cliente-servidor. (10h)\\\xitem TE2: Linguagem de marcação e estilização: HTML e CSS. (10h)\\\xitem TE3: Design Responsivo: CSS 3.0, Media-queries. (10h)\\\xitem TE4: Padronização, acessibilidade na web e Search-Engine-Optimization (SEO). (10h)\\\xitem TE5: Linguagens de scripting: JavaScript. (10h)\\\xitem TE6: Manipulação da página web e controle de eventos: JavaScript e DOM. (10h)} \\\hline
\multicolumn{2}{|p{13cm}|}{Resultados de Aprendizagem} \\\hline
\multicolumn{2}{|p{13cm}|}{\xitem RA1: Interpretar o contexto de uso de aplicações web na Internet, de acordo com as tecnologias fundacionais da Web e autorregulação (TE1).\\\xitem RA2: Modelar estruturas de aplicações web com HTML e CSS, de forma autorregulada (TE2).\\\xitem RA3: Projetar layout de aplicações web com Design Responsivo e autorregulação (TE3).\\\xitem RA4: Implementar aplicações web de acordo com padrões de codificação, regras de design responsivo, acessibilidade na web e otimização de motores de busca, para diferentes plataformas (TE3 e TE4).\\\xitem RA5: Projetar aplicações web utilizando JavaScript, DOM e eventos com APIs de persistência de dados no navegador web e controle de eventos, com objetividade e clareza (TE5 e TE6).} \\\hline

	\end{tabular}
\end{quadro}
\begin{quadro}[ht!]
  \centering
\caption{Unidade Curricular Interação Homem Computador}
\label{unit_themes_ras_21}
\begin{tabular}{|p{5cm}|p{8cm}|}\hline
{\cellcolor{blue1} Unidade curricular} & Interação Homem Computador\\\hline
{\cellcolor{blue1} Núcleo} & Específico\\\hline
{\cellcolor{blue1} Carga-horária (Horas)} & 30\\\hline
\multicolumn{2}{|p{13cm}|}{Temas de Estudo} \\\hline
\multicolumn{2}{|p{13cm}|}{\xitem TE1: Usabilidade: interface de usuário, experiência de usuário, habilidades humanas e limitações, processo cognitivo, abordagens teóricas. (10h)\\\xitem TE2: Princípios e diretrizes de usabilidade: metáforas, estilos e paradigmas de interação, padrões e guias para o projeto de interação. (10h)\\\xitem TE3: Avaliação de usabilidade: métodos de inspeção e de teste com usuários, planejamento da avaliação, análise e interpretação de resultados. (10h)} \\\hline
\multicolumn{2}{|p{13cm}|}{Resultados de Aprendizagem} \\\hline
\multicolumn{2}{|p{13cm}|}{\xitem RA1: Identificar características de usabilidade no contexto da interface e experiência do usuário, considerando os aspectos humanos e tecnológicos (TE1).\\\xitem RA2: Examinar interfaces de usuário, com base nos princípios e diretrizes de usabilidade de forma crítica, científica, criativa e adaptativa às novas tecnologias (TE2).\\\xitem RA3: Planejar avaliações de interfaces de usuário, empregando princípios e diretrizes de usabilidade, métodos de inspeção e de teste com usuários e analisando os resultados com autonomia, responsabilidade e ética (TE2, TE3).} \\\hline

	\end{tabular}
\end{quadro}
\begin{quadro}[ht!]
  \centering
\caption{Unidade Curricular Empreendedorismo}
\label{unit_themes_ras_22}
\begin{tabular}{|p{5cm}|p{8cm}|}\hline
{\cellcolor{blue1} Unidade curricular} & Empreendedorismo\\\hline
{\cellcolor{blue1} Núcleo} & Básico\\\hline
{\cellcolor{blue1} Carga-horária (Horas)} & 30\\\hline
\multicolumn{2}{|p{13cm}|}{Temas de Estudo} \\\hline
\multicolumn{2}{|p{13cm}|}{\xitem TE1: O Mercado e as Oportunidades de Negócios: estrutura de mercado, tendências de novos negócios. (8h)\\\xitem TE2: O empreendedor e fatores de sucesso: características do empreendedor, inovação e criatividade. (10h)\\\xitem TE3: Gestão organizacional : conceitos, ferramentas e desenvolvimento de plano operacional, financeiro e mercadológico de um produto/serviço. (12h) } \\\hline
\multicolumn{2}{|p{13cm}|}{Resultados de Aprendizagem} \\\hline
\multicolumn{2}{|p{13cm}|}{\xitem RA1: Analisar os tipos de mercados existentes e oportunidades de negócios, aplicando ferramentas de gestão de forma economicamente sustentável (TE1).\\\xitem RA2: Aplicar ferramentas com base nas características empreendedoras de forma estratégica na obtenção dos fatores de sucesso (TE2).} \\\hline

	\end{tabular}
\end{quadro}
\begin{quadro}[ht!]
  \centering
\caption{Unidade Curricular Programação Web Back-end}
\label{unit_themes_ras_23}
\begin{tabular}{|p{5cm}|p{8cm}|}\hline
{\cellcolor{blue1} Unidade curricular} & Programação Web Back-end\\\hline
{\cellcolor{blue1} Núcleo} & Específico\\\hline
{\cellcolor{blue1} Carga-horária (Horas)} & 60\\\hline
\multicolumn{2}{|p{13cm}|}{Temas de Estudo} \\\hline
\multicolumn{2}{|p{13cm}|}{\xitem TE1: HTTP: arquitetura cliente-servidor, definições, requisição/resposta, cabeçalhos, parâmetros, CGI. (8h)\\\xitem TE2: Programação no servidor: recebimento de requisições, envio de respostas, tratamento de parâmetros, cookies/sessões. (16h)\\\xitem TE3: Tratamento de exceções: mecanismo de exceções, tipos, captura/lançamento, criar classes, sistema de registro. (8h)\\\xitem TE4: Manipulação de arquivos: sistema de arquivos, tipos, formatos e operações. (8h)\\\xitem TE5: Banco de dados: conexão, APIs, operações, mapeamento objeto-relacional e aplicação. (20h)} \\\hline
\multicolumn{2}{|p{13cm}|}{Resultados de Aprendizagem} \\\hline
\multicolumn{2}{|p{13cm}|}{\xitem RA1: Projetar aplicações web que utilizem o protocolo HTTP e servidores web com conteúdo dinâmico, com eficácia e clareza. (TE1).\\\xitem RA2: Implementar aplicações web que recebam e interpretem parâmetros, utilizem cookies/sessões/tokens para armazenar informações do usuário, com qualidade e robustez. (TE2).\\\xitem RA3: Aplicar mecanismos para tratar exceções em aplicações web de maneira autônoma e eficiente. (TE3)\\\xitem RA4: Projetar sistemas de registro e armazenamento de arquivos de log em aplicações web de maneira autônoma e eficiente. (TE4)\\\xitem RA5: Atuar em projetos implementando mecanismos de persistência com o banco de dados, integrando as funcionalidades de uma aplicação web, utilizando métodos e ferramentas de engenharia de software. (TE5).} \\\hline

	\end{tabular}
\end{quadro}
\begin{quadro}[ht!]
  \centering
\caption{Unidade Curricular Qualidade de Software}
\label{unit_themes_ras_24}
\begin{tabular}{|p{5cm}|p{8cm}|}\hline
{\cellcolor{blue1} Unidade curricular} & Qualidade de Software\\\hline
{\cellcolor{blue1} Núcleo} & Específico\\\hline
{\cellcolor{blue1} Carga-horária (Horas)} & 60\\\hline
\multicolumn{2}{|p{13cm}|}{Temas de Estudo} \\\hline
\multicolumn{2}{|p{13cm}|}{\xitem TE1: Conceitos de qualidade: qualidade de produto/processo, controle, garantia e custo. (10h)\\\xitem TE2: Qualidade de Processo: maturidade de processo, melhoria contínua do processo, normas e padrões, ISO, CMMI, MPS.BR.  (30h)\\\xitem TE3: Qualidade de Produto: produto de software, normas e padrões. (20h)} \\\hline
\multicolumn{2}{|p{13cm}|}{Resultados de Aprendizagem} \\\hline
\multicolumn{2}{|p{13cm}|}{\xitem RA1: Compreender os conceitos de qualidade de produto/processo, técnicas de controle, garantia e o custo associado à atividade de garantia de qualidade. (TE1)\\\xitem RA2: Analisar modelos de maturidade de processo, sua aplicabilidade e custo de implementação. (TE2)\\\xitem RA3: Avaliar a qualidade de produtos de software de acordo com normas e padrões, considerando diferentes atributos de qualidade e utilizando métricas (TE3)} \\\hline

	\end{tabular}
\end{quadro}
\begin{quadro}[ht!]
  \centering
\caption{Unidade Curricular Gerenciamento de Projeto de Software}
\label{unit_themes_ras_25}
\begin{tabular}{|p{5cm}|p{8cm}|}\hline
{\cellcolor{blue1} Unidade curricular} & Gerenciamento de Projeto de Software\\\hline
{\cellcolor{blue1} Núcleo} & Específico\\\hline
{\cellcolor{blue1} Carga-horária (Horas)} & 60\\\hline
\multicolumn{2}{|p{13cm}|}{Temas de Estudo} \\\hline
\multicolumn{2}{|p{13cm}|}{\xitem TE1: Projetos de Software: introdução, fases e PMBOK (5h)\\\xitem TE2: Áreas de conhecimento do PMBOK: escopo, custo risco e integração (10h)\\\xitem TE3: Métricas: estimativas de produto e projeto (10h)\\\xitem TE4: Gerenciamento de Projeto: processo primário e de apoio, ciclo de vida, melhoria de processo (15h)\\\xitem TE5: Desenvolvimento de Projeto: técnicas, ferramentas e execução (20h)} \\\hline
\multicolumn{2}{|p{13cm}|}{Resultados de Aprendizagem} \\\hline
\multicolumn{2}{|p{13cm}|}{\xitem RA1: Utilizar ferramentas e técnicas de gerenciamento de projetos nas fases de desenvolvimento de um projeto, abrangendo as áreas de conhecimento do PMBOK (TE1, TE2 e TE4).\\\xitem RA2: Especificar métricas de estimativas de produto e do projeto e técnicas para acompanhamento do desenvolvimento de software (TE3 e TE5).} \\\hline

	\end{tabular}
\end{quadro}
\begin{quadro}[ht!]
  \centering
\caption{Unidade Curricular Teoria da Computação}
\label{unit_themes_ras_26}
\begin{tabular}{|p{5cm}|p{8cm}|}\hline
{\cellcolor{blue1} Unidade curricular} & Teoria da Computação\\\hline
{\cellcolor{blue1} Núcleo} & Básico\\\hline
{\cellcolor{blue1} Carga-horária (Horas)} & 60\\\hline
\multicolumn{2}{|p{13cm}|}{Temas de Estudo} \\\hline
\multicolumn{2}{|p{13cm}|}{\xitem TE1: Linguagens formais e autômatos: autômatos de estado finito, linguagens regulares. (10h)\\\xitem TE2: Linguagens formais e gramáticas: geração de linguagens, forma normal de Chomsky. (10h)\\\xitem TE3: Máquinas de Turing: princípios, computabilidade e modelos equivalentes. (20h)\\\xitem TE4: Complexidade computacional: complexidade de código-fonte e problemas P e NP. (20h)} \\\hline
\multicolumn{2}{|p{13cm}|}{Resultados de Aprendizagem} \\\hline
\multicolumn{2}{|p{13cm}|}{\xitem RA1: Compreender a noção de computação, a tese de Turing e suas consequências ao estudo da computabilidade efetiva (TE1, TE3)\\\xitem RA2: Identificar características de linguagens de programação como gramática, alfabeto e cadeias, ambiguidade e formas normais (TE2)\\\xitem RA3: Modelar uma máquina de Turing para resolver problemas computacionais (TE3)\\\xitem RA4: Examinar complexidade de algoritmos para validação de requisitos de desempenho de programas computacionais (TE4)} \\\hline

	\end{tabular}
\end{quadro}
\begin{quadro}[ht!]
  \centering
\caption{Unidade Curricular Segurança da Informação}
\label{unit_themes_ras_27}
\begin{tabular}{|p{5cm}|p{8cm}|}\hline
{\cellcolor{blue1} Unidade curricular} & Segurança da Informação\\\hline
{\cellcolor{blue1} Núcleo} & Específico\\\hline
{\cellcolor{blue1} Carga-horária (Horas)} & 60\\\hline
\multicolumn{2}{|p{13cm}|}{Temas de Estudo} \\\hline
\multicolumn{2}{|p{13cm}|}{\xitem TE1: Fundamentos de segurança: políticas, modelos de ameaça e mecanismos de segurança. (10 horas)\\\xitem TE2: Criptologia: sistemas criptográficos simétricos, assimétricos, funções de hash e suas aplicações. (15 horas)\\\xitem TE3: Identidade Digital: certificados digitais, assinatura digital, algoritmos de assinatura e aplicações. (10 horas)\\\xitem TE4: Segurança em redes: protocolos, ameaças, ataques e mecanismos de segurança nas camadas de enlace de dados e transporte do modelo OSI. (15 horas)\\\xitem TE5: Segurança de software: ataques e vulnerabilidades em aplicações, boas práticas de programação e ferramentas de teste de segurança de software (10 horas).} \\\hline
\multicolumn{2}{|p{13cm}|}{Resultados de Aprendizagem} \\\hline
\multicolumn{2}{|p{13cm}|}{\xitem RA1: Compreender as dimensões do processo de desenvolvimento de software que envolvem poíticas de segurança, modelos de ameaça e mecanismos, considerando a responsabilidade do Engenheiro de Software. (TE1)\\\xitem RA2: Projetar sistemas computacionais que façam uso de recursos de criptografia e certificados digitais em sistemas computacionais de forma eficaz. (TE2 e TE3)\\\xitem RA3: Compreender aspectos de segurança em redes, de acordo com a ética e responsabilidade da profissão. (TE4)\\\xitem RA4: Implementar boas práticas de programação, considerando a segurança dos sistemas com autonomia e responsabilidade. (TE5)} \\\hline

	\end{tabular}
\end{quadro}
\begin{quadro}[ht!]
  \centering
\caption{Unidade Curricular Arquitetura de Software}
\label{unit_themes_ras_28}
\begin{tabular}{|p{5cm}|p{8cm}|}\hline
{\cellcolor{blue1} Unidade curricular} & Arquitetura de Software\\\hline
{\cellcolor{blue1} Núcleo} & Específico\\\hline
{\cellcolor{blue1} Carga-horária (Horas)} & 60\\\hline
\multicolumn{2}{|p{13cm}|}{Temas de Estudo} \\\hline
\multicolumn{2}{|p{13cm}|}{\xitem TE1: Orientação a objeto: bibliotecas de orientação a objeto, classes, métodos, acoplamento, coesão, manutenabilidade, métricas e aplicações. (15h)\\\xitem TE2: Bibliotecas e padrões: bibliotecas para organização de código, padrões de projeto e visualização da arquitetura. (15h)\\\xitem TE3: Arquitetura web: camadas, padrões web e REST. (15h)\\\xitem TE4: Estilos arquiteturais: nuvem (escalabilidade), eventos, filas de mensagens e microserviços. (15h) } \\\hline
\multicolumn{2}{|p{13cm}|}{Resultados de Aprendizagem} \\\hline
\multicolumn{2}{|p{13cm}|}{\xitem RA1: Identificar o nível de acoplamento, coesão e manutenabilidade de projetos de software orientado a objetos, considerando diferentes ferramentas, linguagens e tecnologias (TE1)\\\xitem RA2: Entender os cenários nos quais técnicas de reuso de bibliotecas e padrões favorecem a arquitetura de projetos de software orientado a objetos, de forma adaptativa às novas tecnologias (TE2)\\\xitem RA3: Refatorar a arquitetura de aplicações para uso de camadas, padrões web e REST, com qualidade e robustez para diferentes plataformas (TE3).\\\xitem RA4: Entender os benefícios e as mudanças necessárias no uso de nuvem, eventos, filas de mensagens e microserviços, com objetividade e clareza (TE4).} \\\hline

	\end{tabular}
\end{quadro}
\begin{quadro}[ht!]
  \centering
\caption{Unidade Curricular Gerência de Configuração e Manutenção de Software}
\label{unit_themes_ras_29}
\begin{tabular}{|p{5cm}|p{8cm}|}\hline
{\cellcolor{blue1} Unidade curricular} & Gerência de Configuração e Manutenção de Software\\\hline
{\cellcolor{blue1} Núcleo} & Específico\\\hline
{\cellcolor{blue1} Carga-horária (Horas)} & 60\\\hline
\multicolumn{2}{|p{13cm}|}{Temas de Estudo} \\\hline
\multicolumn{2}{|p{13cm}|}{\xitem TE1: Gerenciamento de configuração: conceitos, terminologias, ítens de configuração, armazenamento, controle de mudanças, relatório de status, controle de versões, referenciais (linha de base) e ferramentas. (20h)\\\xitem TE2: Práticas de gerência de configuração: scripts de construção de software, gerênciamento de dependências do software. gerenciamento de ambiente de desenvolvimento, integração/entrega contínua. (32h)\\\xitem TE3: Manutenção de Software: evolução, reengenharia e engenharia reversa. (8h)} \\\hline
\multicolumn{2}{|p{13cm}|}{Resultados de Aprendizagem} \\\hline
\multicolumn{2}{|p{13cm}|}{\xitem RA1: Utilizar ferramentas de gerencionamento de versões e controle de mudança alinhados com um processo de gerência de configuração com ítens de configuração, armazenamento de histórico de alterações e referenciais (TE1)\\\xitem RA2: Configurar um pipeline de entrega contínua com tarefas de construção de software, resolução de dependências e configuração de ambientes de desenvolvimento e produção. (TE2)\\\xitem RA3: Refatorar um sistema legado, utilizando técnicas de reengenharia e engenharia reversa (TE3)} \\\hline

	\end{tabular}
\end{quadro}
\begin{quadro}[ht!]
  \centering
\caption{Unidade Curricular Teste de Software }
\label{unit_themes_ras_30}
\begin{tabular}{|p{5cm}|p{8cm}|}\hline
{\cellcolor{blue1} Unidade curricular} & Teste de Software \\\hline
{\cellcolor{blue1} Núcleo} & Específico\\\hline
{\cellcolor{blue1} Carga-horária (Horas)} & 60\\\hline
\multicolumn{2}{|p{13cm}|}{Temas de Estudo} \\\hline
\multicolumn{2}{|p{13cm}|}{\xitem TE1: Fundamentos de teste de software: Verificação/Validação/Teste, terminologia de defeitos/erros/falhas, casos de teste, etapas do teste e limitações. (10h)\\\xitem TE2: Testes automatizados: Estratégias de teste e Testes de unidade (mocks, stubs, assertivas), integração e sistema. (15h)\\\xitem TE3: Técnicas de geração de casos de teste: critérios de teste caixa-preta (particionamento em classes de equivalência, valor limite, tabela de decisão) e caixa-branca (fluxo de controle, fluxo de dados, cobertura de comandos, decisões, caminho básico, caminhos independentes). (20h)\\\xitem TE4: Teste end-to-end: seletores de interface, ferramentas de automatização, Integração contínua e teste baseado em modelo. (15h)} \\\hline
\multicolumn{2}{|p{13cm}|}{Resultados de Aprendizagem} \\\hline
\multicolumn{2}{|p{13cm}|}{\xitem RA1: Implementar scripts de testes de unidade, integração e sistema, considerando as boas práticas de elaboração de casos de teste, a qualidade e robustez das diferentes plataformas (TE1 e TE2).\\\xitem RA2: Aplicar técnicas de seleção e adequação de teste caixa-branca e caixa-preta, com eficácia, objetividade e clareza (TE3).\\\xitem RA3: Implementar scripts de teste end-to-end e aplicação de técnicas de teste baseado em modelo, considerando as linguagens de programação e diferentes plataformas utilizadas (TE4).} \\\hline

	\end{tabular}
\end{quadro}
\begin{quadro}[ht!]
  \centering
\caption{Unidade Curricular Metodologia de Pesquisa}
\label{unit_themes_ras_31}
\begin{tabular}{|p{5cm}|p{8cm}|}\hline
{\cellcolor{blue1} Unidade curricular} & Metodologia de Pesquisa\\\hline
{\cellcolor{blue1} Núcleo} & Básico\\\hline
{\cellcolor{blue1} Carga-horária (Horas)} & 60\\\hline
\multicolumn{2}{|p{13cm}|}{Temas de Estudo} \\\hline
\multicolumn{2}{|p{13cm}|}{\xitem TE1: Fundamentos da metodologia científica: ciência e tecnologia, pesquisa e desenvolvimento tecnológico, inovação tecnológica, tipos de trabalhos acadêmicos e níveis de trabalhos de conclusão (TCC de graduação e especialização, dissertação e tese), processo de produção e comunicação científica (projeto de pesquisa, estrutura do projeto de pesquisa), estrutura de um artigo científico, comparativo entre artigo científico e trabalho de conclusão (8h)\\\xitem TE2: Planejamento e desenvolvimento da pesquisa: problema de pesquisa, objetivos e hipóteses de pesquisa (8h)\\\xitem TE3: Elaboração do referencial teórico: busca em bases de dados bibliográficas, técnicas de revisão da literatura (revisão bibliométrica, revisão e mapeamento sistemático), ferramentas de apoio, normas ABNT de citações e referências e escrita científica (10h)\\\xitem TE4: Tipos de pesquisa: classificação das pesquisas e métodos de pesquisa, survey, estudo de caso, pesquisa-ação, pesquisa experimental (experimento e quasi-experimento) (20h)\\\xitem TE5: Elaboração do trabalho de conclusão de curso: revisão bibliográfica, definição de tema e objetivos (geral e específicos) e comunicação científica (refinamento do trabalho acadêmico) (14h)} \\\hline
\multicolumn{2}{|p{13cm}|}{Resultados de Aprendizagem} \\\hline
\multicolumn{2}{|p{13cm}|}{\xitem RA1: Reavaliar os requisitos para implementação de forma eficiente considerando a metodologia adequada, os procedimentos e as etapas da pesquisa prévia necessária (TE1).\\\xitem RA2: Documentar com rigor técnico e científico o processo de desenvolvimento de sistemas de software e hardware (TE2 e TE3).\\\xitem RA3: Definir métodos de pesquisa para avaliação de metodologias e funcionalidades das soluções de software e hardware (TE4).\\\xitem RA4: Elaborar projeto e testes para validação de soluções de software e hardware com rigor técnico (TE5).} \\\hline

	\end{tabular}
\end{quadro}
\begin{quadro}[ht!]
  \centering
\caption{Unidade Curricular Estratégias de Inovação}
\label{unit_themes_ras_32}
\begin{tabular}{|p{5cm}|p{8cm}|}\hline
{\cellcolor{blue1} Unidade curricular} & Estratégias de Inovação\\\hline
{\cellcolor{blue1} Núcleo} & Básico\\\hline
{\cellcolor{blue1} Carga-horária (Horas)} & 30\\\hline
\multicolumn{2}{|p{13cm}|}{Temas de Estudo} \\\hline
\multicolumn{2}{|p{13cm}|}{\xitem TE1: Fundamentos de Inovação: Conceitos de Estratégias empresariais, Tipos e graus de inovação. (10h)\\\xitem TE2: Estratégias de Inovação: culturas e ambientes de inovação, Cases de estratégia de Inovação e aplicações. (20h)} \\\hline
\multicolumn{2}{|p{13cm}|}{Resultados de Aprendizagem} \\\hline
\multicolumn{2}{|p{13cm}|}{\xitem RA1: Identificar do ambiente competitivo das organizações e a importância do desenvolvimento da inovação, demonstrando a compreensão dos tipos e graus de inovação (TE1).\\\xitem RA2: Analisar criticamente estratégias de inovação descrevendo a cultura e o ambiente, propondo a aplicação de conceitos de inovação em uma organização com autonomia e responsabilidade (TE2).} \\\hline

	\end{tabular}
\end{quadro}
\begin{quadro}[ht!]
  \centering
\caption{Unidade Curricular Oficina de Integração 1}
\label{unit_themes_ras_33}
\begin{tabular}{|p{5cm}|p{8cm}|}\hline
{\cellcolor{blue1} Unidade curricular} & Oficina de Integração 1\\\hline
{\cellcolor{blue1} Núcleo} & Específico\\\hline
{\cellcolor{blue1} Carga-horária (Horas)} & 120\\\hline
\multicolumn{2}{|p{13cm}|}{Temas de Estudo} \\\hline
\multicolumn{2}{|p{13cm}|}{\xitem TE1: Integração dos conhecimentos das disciplinas do primeiro, segundo e terceiro, quarto e quinto períodos. Aplicação desses conhecimentos em todas as etapas do desenvolvimento de um sistema computacional que resolva um problema real da comunidade externa a UTFPR. (120h)} \\\hline
\multicolumn{2}{|p{13cm}|}{Resultados de Aprendizagem} \\\hline
\multicolumn{2}{|p{13cm}|}{\xitem RA1: Integrar os conhecimentos das disciplinas em todo o curso. Aplicação desses conhecimentos em todas as etapas do desenvolvimento de um sistema computacional (TE1).} \\\hline

	\end{tabular}
\end{quadro}
\begin{quadro}[ht!]
  \centering
\caption{Unidade Curricular Programação para Dispositivos Móveis}
\label{unit_themes_ras_34}
\begin{tabular}{|p{5cm}|p{8cm}|}\hline
{\cellcolor{blue1} Unidade curricular} & Programação para Dispositivos Móveis\\\hline
{\cellcolor{blue1} Núcleo} & Profissionalizante\\\hline
{\cellcolor{blue1} Carga-horária (Horas)} & 60\\\hline
\multicolumn{2}{|p{13cm}|}{Temas de Estudo} \\\hline
\multicolumn{2}{|p{13cm}|}{\xitem TE1: Dispositivos móveis e tecnologias de desenvolvimento de aplicativos móveis: histórico, mercado, plataformas existentes e segurança (5h)\\\xitem TE2: Framework para desenvolvimento: características, ferramentas de desenvolvimento e aplicações (5h)\\\xitem TE3: Projeto de interfaces: recursos tecnológicos, componentes de interface, navegação, posicionamento e layout (20h)\\\xitem TE4: Persistência de dados: definição, tipos de armazenamento e manipulação de dados (20h)\\\xitem TE5: Recursos da plataforma e controle de permissões: sensores (câmera, GPS, acelerômetro), mapas e notificações (10h)} \\\hline
\multicolumn{2}{|p{13cm}|}{Resultados de Aprendizagem} \\\hline
\multicolumn{2}{|p{13cm}|}{\xitem RA1: Utilizar frameworks para desenvolvimento de dispositivos móveis, entendendo as características e as limitações existentes nas diferentes plataformas com objetividade e clareza (TE1 e TE2).\\\xitem RA2: Projetar interfaces com componentes, navegação e diferentes tipos de posicionamento e layout utilizando ferramentas para desenvolvimento de aplicações para dispositivos móveis com eficácia (TE3).\\\xitem RA3: Implementar aplicações para dispositivos móveis utilizando APIs de persistência de dados e recursos da plataforma, como sensores, mapas e notificações com qualidade e robustez (TE4 e TE5).} \\\hline

	\end{tabular}
\end{quadro}
\begin{quadro}[ht!]
  \centering
\caption{Unidade Curricular Programação Web Fullstack}
\label{unit_themes_ras_35}
\begin{tabular}{|p{5cm}|p{8cm}|}\hline
{\cellcolor{blue1} Unidade curricular} & Programação Web Fullstack\\\hline
{\cellcolor{blue1} Núcleo} & Profissionalizante\\\hline
{\cellcolor{blue1} Carga-horária (Horas)} & 60\\\hline
\multicolumn{2}{|p{13cm}|}{Temas de Estudo} \\\hline
\multicolumn{2}{|p{13cm}|}{\xitem TE1: Frameworks de desenvolvimento no cliente: arquitetura cliente-servidor, DOM HTML, HTML5, AJAX, Webservices, Web APIs, JSON, Web components. (25h)\\\xitem TE2: Frameworks de desenvolvimento no servidor: programação assíncrona, webservices/REST, estratégias de cache, características dos frameworks web e aplicações. (15h)\\\xitem TE3: Segurança em aplicações web: criptografia, XSS, SQL Inject, HTTPS, chaves assimétricas e certificados digitais. (10h)\\\xitem TE4: Projeto de aplicação web fullstack: comunicação cliente-servidor utilizando frameworks de alto nível, disponibilização do cliente e servidor na Web e integração com banco de dados. (10h)} \\\hline
\multicolumn{2}{|p{13cm}|}{Resultados de Aprendizagem} \\\hline
\multicolumn{2}{|p{13cm}|}{\xitem RA1: Desenvolver interfaces web utilizando frameworks de front-end, considerando a qualidade da aplicação web (TE1)\\\xitem RA2: Desenvolver Web APIs utilizando frameworks de back-end, com persistência de dados, escalabilidade e portabilidade das soluções (TE2)\\\xitem RA3: Aplicar frameworks de desenvolvimento web para projetar soluções com autonomia, responsabilidade e ética, considerando a segurança da aplicação web (TE3)\\\xitem RA4: Desenvolver projetos de aplicações web fullstack com frameworks, integrando o front-end e back-end, de forma colaborativa utilizando metodologias de gerenciamento de projeto (TE4)} \\\hline

	\end{tabular}
\end{quadro}
\begin{quadro}[ht!]
  \centering
\caption{Unidade Curricular Oficina de Integração 2}
\label{unit_themes_ras_36}
\begin{tabular}{|p{5cm}|p{8cm}|}\hline
{\cellcolor{blue1} Unidade curricular} & Oficina de Integração 2\\\hline
{\cellcolor{blue1} Núcleo} & Específico\\\hline
{\cellcolor{blue1} Carga-horária (Horas)} & 210\\\hline
\multicolumn{2}{|p{13cm}|}{Temas de Estudo} \\\hline
\multicolumn{2}{|p{13cm}|}{\xitem TE1: Integração dos conhecimentos das disciplinas em todo o curso. Aplicação desses conhecimentos em todas as etapas do desenvolvimento de um sistema computacional para resolver um problema da comunidade externa a UTFPR. (210h)} \\\hline
\multicolumn{2}{|p{13cm}|}{Resultados de Aprendizagem} \\\hline
\multicolumn{2}{|p{13cm}|}{\xitem RA1: Integrar os conhecimentos das disciplinas em todo o curso. Aplicação desses conhecimentos em todas as etapas do desenvolvimento de um sistema computacional (TE1).} \\\hline

	\end{tabular}
\end{quadro}